\documentclass{beamer}\usepackage[]{graphicx}\usepackage[]{color}
%% maxwidth is the original width if it is less than linewidth
%% otherwise use linewidth (to make sure the graphics do not exceed the margin)
\makeatletter
\def\maxwidth{ %
  \ifdim\Gin@nat@width>\linewidth
    \linewidth
  \else
    \Gin@nat@width
  \fi
}
\makeatother

\definecolor{fgcolor}{rgb}{0.345, 0.345, 0.345}
\newcommand{\hlnum}[1]{\textcolor[rgb]{0.686,0.059,0.569}{#1}}%
\newcommand{\hlstr}[1]{\textcolor[rgb]{0.192,0.494,0.8}{#1}}%
\newcommand{\hlcom}[1]{\textcolor[rgb]{0.678,0.584,0.686}{\textit{#1}}}%
\newcommand{\hlopt}[1]{\textcolor[rgb]{0,0,0}{#1}}%
\newcommand{\hlstd}[1]{\textcolor[rgb]{0.345,0.345,0.345}{#1}}%
\newcommand{\hlkwa}[1]{\textcolor[rgb]{0.161,0.373,0.58}{\textbf{#1}}}%
\newcommand{\hlkwb}[1]{\textcolor[rgb]{0.69,0.353,0.396}{#1}}%
\newcommand{\hlkwc}[1]{\textcolor[rgb]{0.333,0.667,0.333}{#1}}%
\newcommand{\hlkwd}[1]{\textcolor[rgb]{0.737,0.353,0.396}{\textbf{#1}}}%
\let\hlipl\hlkwb

\usepackage{framed}
\makeatletter
\newenvironment{kframe}{%
 \def\at@end@of@kframe{}%
 \ifinner\ifhmode%
  \def\at@end@of@kframe{\end{minipage}}%
  \begin{minipage}{\columnwidth}%
 \fi\fi%
 \def\FrameCommand##1{\hskip\@totalleftmargin \hskip-\fboxsep
 \colorbox{shadecolor}{##1}\hskip-\fboxsep
     % There is no \\@totalrightmargin, so:
     \hskip-\linewidth \hskip-\@totalleftmargin \hskip\columnwidth}%
 \MakeFramed {\advance\hsize-\width
   \@totalleftmargin\z@ \linewidth\hsize
   \@setminipage}}%
 {\par\unskip\endMakeFramed%
 \at@end@of@kframe}
\makeatother

\definecolor{shadecolor}{rgb}{.97, .97, .97}
\definecolor{messagecolor}{rgb}{0, 0, 0}
\definecolor{warningcolor}{rgb}{1, 0, 1}
\definecolor{errorcolor}{rgb}{1, 0, 0}
\newenvironment{knitrout}{}{} % an empty environment to be redefined in TeX

\usepackage{alltt}
\usepackage{listings}
\IfFileExists{upquote.sty}{\usepackage{upquote}}{}
\begin{document}

\title{The 9522: A python project}
\author{Heidi Beezub}

\begin{frame}
  \titlepage
\end{frame}

\begin{frame}
  \frametitle{Outline}
    \tableofcontents
\end{frame}

\section{Install and Load Libraries}
\begin{frame}[fragile]
  \frametitle{Install and Load Libraries}
    \begin{itemize}
      \item<1->
\begin{knitrout}
\definecolor{shadecolor}{rgb}{0.969, 0.969, 0.969}\color{fgcolor}\begin{kframe}
\begin{alltt}
\hlkwd{library}\hlstd{(dplyr)}
\end{alltt}
\end{kframe}
\end{knitrout}
      \item<2->
\begin{knitrout}
\definecolor{shadecolor}{rgb}{0.969, 0.969, 0.969}\color{fgcolor}\begin{kframe}
\begin{alltt}
\hlkwd{library}\hlstd{(tidytext)}
\end{alltt}
\end{kframe}
\end{knitrout}
    \item<3->
\begin{knitrout}
\definecolor{shadecolor}{rgb}{0.969, 0.969, 0.969}\color{fgcolor}\begin{kframe}
\begin{alltt}
\hlkwd{library}\hlstd{(gutenbergr)}
\end{alltt}
\end{kframe}
\end{knitrout}
    \item<4->
\begin{knitrout}
\definecolor{shadecolor}{rgb}{0.969, 0.969, 0.969}\color{fgcolor}\begin{kframe}
\begin{alltt}
\hlkwd{library}\hlstd{(PythonInR)}
\end{alltt}
\end{kframe}
\end{knitrout}
    \item<5->
\begin{knitrout}
\definecolor{shadecolor}{rgb}{0.969, 0.969, 0.969}\color{fgcolor}\begin{kframe}
\begin{alltt}
\hlkwd{library}\hlstd{(stringr)}
\end{alltt}
\end{kframe}
\end{knitrout}
    
    \end{itemize}
\end{frame}

\section{Background}
\begin{frame}[fragile]
  \frametitle{The 9522 and Rejects}
  The DCM 9522 report is one of the reports I use in order to correct 'rejects' in my job.DCM stand for Department Cost Manager.  It includes multiple departments and all the products (both supplies & services) that each department provides.
  
  
\textbf{What is a reject?}

   A reject is data that the system collects but then cannot process or put into the correct category.  Rejects are typically a 'new' item that the simply needs the interface created or built in the system.  Sometimes this is an existing item that is either coming into the system in a new way.

\end{frame}

\section{Code for IP Number}
\begin{frame}[fragile]
  \frametitle{Code for IP Number}
  \lstinputlisting[language=Python]{9522_slide_IPNUM_only.py}
   
\begin{knitrout}
\definecolor{shadecolor}{rgb}{0.969, 0.969, 0.969}\color{fgcolor}\begin{kframe}
\begin{alltt}
def \hlkwd{process_file}(reader):
    result_line= \hlstr{''}
    result=\hlstr{''}
    slide=\hlstr{"IPNUM"} +\hlstr{'\textbackslash{}n'}
\hlcom{    #first we need to add headers}
    with \hlkwd{open}(\hlstr{'9522_slide_new.csv'}, \hlstr{'a'}) as output_file:
        \hlkwd{output_file.write}(\hlstr{'\hlstr{"IPNUM"}'} +\hlstr{'\textbackslash{}n'})
    for line in reader:
        line=\hlkwd{line.strip}()    \hlcom{#removes leading/trailing whitespace}
        field = \hlkwd{line.split}()

        if \hlkwd{len}(field)>3:
            for i in \hlkwd{range}(0,2):
\hlcom{                #find IPNUM}
                if field[i] == \hlstr{'IP'} and field[i+1] ==\hlstr{'NUM'}:
\hlcom{                    #save IPNUM}
                    ipnum=field[i+2]\hlkwd{.strip}(\hlstr{':'})
\hlcom{                        #(last item before write to file)}
\hlcom{                        #when result has no data it is a blank line}
                    if result_line != None:
                        with \hlkwd{open}(\hlstr{'9522_slide_new.csv'}, \hlstr{'a'}) as output_file:
                            list_2_line=ipnum
                            slide=slide+list_2_line+\hlstr{'\textbackslash{}n'}
\hlcom{                            #print('list 2 line=',list_2_line)}
                            \hlkwd{output_file.write}(list_2_line+\hlstr{'\textbackslash{}n'})
                            result=\hlstr{''}
                            list_2_line=\hlstr{''}
    \hlkwd{print}(slide)       
if __name__ == \hlstr{'__main__'}:
    with \hlkwd{open}(\hlstr{'9522_slide.txt'}, \hlstr{'r'}) as input_file:
        \hlkwd{process_file}(input_file)
\end{alltt}


{\ttfamily\noindent\bfseries\color{errorcolor}{\#\# Error: <text>:1:5: unexpected symbol\\\#\# 1: def process\_file\\\#\#\ \ \ \ \ \ \ \  \textasciicircum{}}}\end{kframe}
\end{knitrout}


   
   
   
\end{frame}
%% Is this a comment?
\section{Unpack the Words}
\begin{frame}[fragile]
  \frametitle{Unpack the Words}
  
    
   
\begin{knitrout}
\definecolor{shadecolor}{rgb}{0.969, 0.969, 0.969}\color{fgcolor}\begin{kframe}
\begin{alltt}
\hlstd{df}\hlkwb{<-}\hlkwd{gutenberg_works}\hlstd{(}\hlkwd{str_detect}\hlstd{(title,}\hlstr{'Dracula'}\hlstd{))}
\hlstd{df}\hlopt{$}\hlstd{gutenberg_id}
\end{alltt}
\begin{verbatim}
## [1]   345 10150
\end{verbatim}
\begin{alltt}
\hlstd{df}\hlopt{$}\hlstd{title}
\end{alltt}
\begin{verbatim}
## [1] "Dracula"         "Dracula's Guest"
\end{verbatim}
\end{kframe}
\end{knitrout}

  
\begin{knitrout}
\definecolor{shadecolor}{rgb}{0.969, 0.969, 0.969}\color{fgcolor}\begin{kframe}
\begin{alltt}
\hlstd{dracula}\hlkwb{<-}\hlkwd{gutenberg_download}\hlstd{(}\hlnum{345}\hlstd{)}
\hlkwd{colnames}\hlstd{(dracula)}
\end{alltt}
\begin{verbatim}
## [1] "gutenberg_id" "text"
\end{verbatim}
\begin{alltt}
\hlkwd{substr}\hlstd{(dracula}\hlopt{$}\hlstd{text[}\hlnum{500}\hlstd{],}\hlnum{1}\hlstd{,}\hlnum{21}\hlstd{)}
\end{alltt}
\begin{verbatim}
## [1] "my own disappointment"
\end{verbatim}
\end{kframe}
\end{knitrout}
 
  
\begin{knitrout}
\definecolor{shadecolor}{rgb}{0.969, 0.969, 0.969}\color{fgcolor}\begin{kframe}
\begin{alltt}
\hlstd{dracula_words}\hlkwb{<-}\hlstd{dracula}\hlopt
  \hlkwd{unnest_tokens}\hlstd{(word,text)}
\hlkwd{colnames}\hlstd{(dracula_words)}
\end{alltt}
\begin{verbatim}
## [1] "gutenberg_id" "word"
\end{verbatim}
\begin{alltt}
\hlstd{dracula_words[}\hlnum{498}\hlopt{:}\hlnum{500}\hlstd{,]}
\end{alltt}
\begin{verbatim}
## # A tibble: 3 x 2
##   gutenberg_id  word
##          <int> <chr>
## 1          345  fail
## 2          345    to
## 3          345  have
\end{verbatim}
\end{kframe}
\end{knitrout}
\end{frame}

\section{The Bing Lexicon}
\begin{frame}[fragile]
  \frametitle{The Bing Lexicon}
\begin{knitrout}
\definecolor{shadecolor}{rgb}{0.969, 0.969, 0.969}\color{fgcolor}\begin{kframe}
\begin{alltt}
  \hlstd{bing}\hlkwb{<-}\hlkwd{get_sentiments}\hlstd{(}\hlstr{'bing'}\hlstd{)}
  \hlkwd{colnames}\hlstd{(bing)}
\end{alltt}
\begin{verbatim}
## [1] "word"      "sentiment"
\end{verbatim}
\begin{alltt}
  \hlstd{bing[}\hlnum{498}\hlopt{:}\hlnum{500}\hlstd{,]}
\end{alltt}
\begin{verbatim}
## # A tibble: 3 x 2
##          word sentiment
##         <chr>     <chr>
## 1     bereave  negative
## 2 bereavement  negative
## 3      bereft  negative
\end{verbatim}
\end{kframe}
\end{knitrout}
\end{frame}

\section{The Inner Join}
\begin{frame}[fragile]
  \frametitle{The Inner Join}
\begin{knitrout}
\definecolor{shadecolor}{rgb}{0.969, 0.969, 0.969}\color{fgcolor}\begin{kframe}
\begin{alltt}
  \hlstd{dracula_words}\hlkwb{<-}\hlkwd{inner_join}\hlstd{(dracula_words,bing)}
\end{alltt}


{\ttfamily\noindent\itshape\color{messagecolor}{\#\# Joining, by = "{}word"{}}}\begin{alltt}
  \hlstd{dracula_words}\hlopt{$}\hlstd{gutenberg_id}\hlkwb{<-}\hlkwa{NULL}
  \hlstd{dracula_words[}\hlnum{498}\hlopt{:}\hlnum{500}\hlstd{,]}
\end{alltt}
\begin{verbatim}
## # A tibble: 3 x 2
##      word sentiment
##     <chr>     <chr>
## 1   great  positive
## 2    love  positive
## 3 crowded  negative
\end{verbatim}
\end{kframe}
\end{knitrout}
\end{frame}

\end{document}
